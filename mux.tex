\documentclass{article}
\usepackage{amssymb}
\usepackage{graphicx} % Required for inserting images
\usepackage{tikz}
\usepackage{amsmath}
\usepackage{amsfonts}
\usepackage{float}
\usepackage{circuitikz}
\usetikzlibrary{circuits.ee.IEC}

\title{Gate Assignment}
\author{AKSHAY KUMAR}
\date{}

\begin{document}
\maketitle

\begin{enumerate}
\item In the circuit shown below, P and Q are the inputs. The logical function realized by the circuit shown below is

\begin{figure}[H]
\centering

\begin{circuitikz}
\draw (0,0) --(8,0)--(8,8)--(0,8)-- cycle;
 % Draw the input lines
 \draw (0,1.6) -- (-2.0,1.6) node [left] {$P$};
\draw (0,1.6)  (0.6,1.6) node [right] {$I_1$};
\draw (0,6.4) -- (-2.0,6.4) to [ground](-2,5);
\draw (-3,5)--(-1,5);
\draw (-2.7,4.8)--(-1.3,4.8);
\draw (-2.4,4.6)--(-1.6,4.6);
\draw (0,6.4)--(0.6,6.4) node [right] {$I_0$};
\draw (3.7,0) (3.7,0.5) node [right] {$Sel$};
\draw(4,0)--(4,-2)--cycle;(-4,-2)--cycle;
\draw(-2,-2)--(4,-2)--cycle;
\draw (-0.8,-2)-- (-2,-2) node [left] {$Q$};
    
 % Draw the output line
\draw (8,4.0) -- (9,4.0) node [right] {Y};
% Draw the multiplexer symbol
\draw (3.7,4.0)  node{$2$};
\draw (4.0,4.0)  node{x};
\draw (4.3,4.0)  node{$1$};
\draw (4.0,3.4)  node{MUX};
\end{circuitikz}
\end{figure}
\begin{enumerate}
    \item Y=PQ
    \item Y=P+Q    
    \item Y= $\overline{PQ}$
    \item Y= $\overline{P+Q}$
\end{enumerate}
\end{enumerate}
\end{document}












